\documentclass[12pt, letterpaper]{article}
\usepackage[brazilian]{babel}
\usepackage{geometry}
\geometry{
	a4paper,
	total={170mm,257mm},
	left=20mm,
	top=20mm,
}
\usepackage{multirow}
\usepackage{csquotes}
\usepackage{biblatex}
\addbibresource{refs.bib}

\title{\textbf{Projeto: Matrizes Esparsas}}
\date{}
\author{Antônio Joabe Alves Morais \\ Iarley Natã Lopes Souza}

\begin{document}
	% Descrição da estrutura: aplicações, motivação
	% Tomada de decisões relaativas à implementação
	% Como o trabalho foi dividido entre a dupla
	% Dificuldades
	% Testes executados
	% Análise de complexidade da insert, get e sum
	% Referências bibliográficas
	\maketitle

	\section{Descrição do Problema} \label{descr}
		Uma matriz esparsa é uma matriz que tem mais valores nulos do que não nulos, ou seja, mais "zerados" do que "setados", ao contrário da matriz densa.

		Quanto as aplicações da matriz esparsa, pode-se citar: \cite{Brownlee2018, UnivespME2016, Cerebras2019}
		\begin{itemize}
			\item Machine Learning;
			\item Codificação de dados (Data Encoding);
			\item Otimização de algoritmos;
			\item Sistemas computacionais baseados em IA;
			\item Entre várias outras aplicações.
		\end{itemize}

		Para representar uma matriz esparsa na programação, diversas técnicas e estruturas de dados podem ser aplicadas, nesse caso, no entanto, usaremos listas simplesmente encadeadas circulares.
	\section{Decisões Tomadas} \label{decis}
		Não fizemos muitas coisas que ficam fora do que foi proposto no documento inicial, o que implementamos foram 3 funções extras na \verb|SparseMatrix.cpp|:
		\begin{itemize}
			\item \verb|void getHead|: \\
				retorna o atributo \verb|head|, declarado na \verb|SparseMatrix.hpp|;
			\item \verb|void getLineQty|: \\
				retorna o número de linhas da matriz;
			\item \verb|void getColQty|: \\
				retorna o número de colunas da matriz.
		\end{itemize}

		Também optamos por implementar uma \verb|main()| interativa, devidamente documentada na seção \ref{tests}.
	\section{Divisão} \label{div}
		A divisão do projeto foi sendo decidida no decorrer do projeto:
		
		Joabe ficou responsável pelas funções \verb|SparseMatrix()|(construtor)\verb|, insert(), print()| e \verb|readSparseMatrix()| \cite{ArquivosCFB2017, CppFilesShmeowlex2021}; pela análise assintótica; e pela escrita deste documento \cite{LearnOverleaf2022}.

		Iarley ficou responsável pelas funções \verb|~SparseMatrix()|(destrutor)\verb|, get(), sum()| e \\ \verb|multiply()|; e pela \verb|main()| interativa.
	\section{Dificuldades} \label{difc}
		Dentre as dificuldades que enfrentamos estão: saber implementar cada função da matriz; pensar em cada possibilidade de erro que uma certa implementação pode gerar; saber como ligar cada nó em diferentes situações; e comunicar de forma concisa e organizada cada caso e conceito, o que pode ser bem confuso.
	\section{Testes Executados} \label{tests}
		Como dito na seção \ref{decis}, foi implementado uma \verb|main()| interativa, que será documentada a seguir:
		\begin{center}
			\begin{tabular}{|p{4cm}||p{8cm}|}
				\hline
				\multicolumn{2}{|c|}{Comandos da Main Interativa} \\ \hline
				\hline
				\centering{\verb|create m n|}     & Cria uma matriz com $m$ linhas e $n$ colunas \\ \hline
				\centering{\verb|insert i j x a|} & Insere um valor $x$ na posição $(i, j)$ na matriz $a$ \\ \hline
				\centering{\verb|get i j a|}      & Retorna um valor inserido na posição $(i, j)$ na matriz $a$ \\ \hline
				\centering{\verb|show a|}         & Mostra no terminal a matriz $a$ \\ \hline
				\centering{\verb|showAll|}        & Mostra no terminal todas as matrizes existentes \\ \hline
				\centering{\verb|sum a b|}        & Soma as matrizes $a$ e $b$ e cria uma matriz resultado \\ \hline
				\centering{\verb|mult a b|}       & Multiplica as matrizes $a$ e $b$ e cria uma matriz resultado \\ \hline
				\centering{\verb|read s|}*        & Lê um arquivo com nome $s$ (uma string, incluindo .txt) e cria uma matriz com os dados desse arquivo. \\ \hline
				\centering{\verb|exit|}           & Desaloca todas as matrizes e encerra o programa \\ \hline
			\end{tabular}
			
		\end{center}
			*Quanto aos testes envolvendo arquivos, já existe um arquivo preenchido na pasta de projeto, chamado $A.txt$. Se o usuário quiser, ele pode modificá-lo e/ou até criar um novo arquivo, só tendo o cuidado de passar o nome correto ao chamar o comando \verb|read|.
			\section{Análise de Complexidade} \label{complx}
			Nessa análise na notação \emph{Big O}, vamos focar somente nas complexidades não constantes ($O(n), O(n^2)$...), pois as linhas com complexidade constante não influenciam no resultado final, além de serem numerosas no programa.
			\begin{itemize}
				\item \verb|get()|:
					\begin{quote}
						Na linha 185, temos um laço \verb|while|, que percorre a lista até encontrar a linha passada por parâmetro. $O(n)$.

						Depois, na linha 190, um laço \verb|while| percorre a lista até encontrar a coluna. $O(n)$.

						\[O(n) + O(n) = 2 \times O(n)\]

						Ignorando as constantes, temos $O(n)$.
					\end{quote}

				\item \verb|insert()|:
					\begin{quote}
						Nessa função, o pior caso vai acontecer quando for preciso desalocar um nó, que demanda percorrer a lista várias vezes para ajustar ponteiros e, finalmente, deletar o nó.

						Primeiramente, temos o primeiro $O(n)$, na linha 94, um laço \verb|for| usado para encontrar a linha desejada.

						Depois vamos para a linha 100, que faz chamada da função \verb|get()|, já analisada anteriormente. $O(n)$.

						Ao entrar nessa condição, vamos para o \verb|else|, da linha 106, já que queremos desalocar, a fim de obter o pior caso.

						Temos na linha 109 um laço while que acha o nó a ser desalocado. $O(n)$.

						Na linha 113, temos um laço \verb|while|, que percorre a matrix até encontrar o nó anterior ao nó a ser desalocado. $O(n)$.

						Na linha 124, um laço \verb|for| encontra a coluna passado por parâmetro. $O(n)$.

						E o laço \verb|while| da linha 129 é análogo ao da linha 113. $O(n)$.

						\[O(n) + O(n) + O(n) + O(n) + O(n) + O(n) = 6 \times O(n)\]

						Portanto, temos $O(n)$.
					\end{quote}

				\item \verb|sum()|:
					\begin{quote}
						Nessa função, ocorre um aninhamento de complexidades não constantes. Por isso, essa análise será um pouco diferente das anteriores. 

						Vamos começar do "núcleo" do primeiro while (linha 103): a linha 106.

						Nela, ocorre a chamada da função \verb|get()|: $O(n)$, dentro da função \verb|insert()|: $O(n)$. O que dá a essa linha uma complexidade $O(n^2)$.

						A linha 106 é executada $n$ vezes, pois está inserida num laço \verb|while|, o que eleva o nível de complexidade para $O(n^3)$.

						E tudo isso está dentro de outro laço \verb|while|, que é executado $n$ vezes, o que eleva a complexidade para $O(n^4)$.

						Depois a linha 113 faz um processo semelhante ao descrito anterior, executando a soma em si, ao contrário do bloco anterior, que executa uma cópia. $O(n^4)$.

						\[ O(n^4) + O(n^4) = 2 \times O(n^4) \]

						Portanto, temos $O(n^4)$.
					\end{quote}

			\end{itemize}
	\pagebreak
	\printbibliography
\end{document}
\documentclass[12pt, letterpaper]{article}
\usepackage[brazilian]{babel}
\usepackage{geometry}
\geometry{
	a4paper,
	total={170mm,257mm},
	left=20mm,
	top=20mm,
}
\usepackage{csquotes}
\usepackage{biblatex}
\addbibresource{refs.bib}

\title{\textbf{Projeto: Matrizes Esparsas}}
\date{}
\author{Antônio Joabe Alves Morais \\ Iarley Natã Souza Lopes}

\begin{document}
	% Descrição da estrutura: aplicações, motivação
	% Tomada de decisões relaativas à implementação
	% Como o trabalho foi dividido entre a dupla
	% Dificuldades
	% Testes executados
	% Análise de complexidade da insert, get e sum
	% Referências bibliográficas
	\maketitle

	\section{Descrição do Problema} \label{desc}
		Uma matriz esparsa é uma matriz que tem mais valores nulos do que não nulos, ou seja, mais "zerados" do que "setados", ao contrário da matriz densa.

		Quanto as aplicações da matriz esparsa, pode-se citar: \cite{Brownlee2018, UnivespME2016, Cerebras2019}
		\begin{itemize}
			\item Machine Learning;
			\item Codificação de dados (Data Encoding);
			\item Otimização de algoritmos;
			\item Sistemas computacionais baseados em IA;
			\item Entre várias outras aplicações.
		\end{itemize}

		Para representar uma matriz esparsa na programação, diversas técnicas e estruturas de dados podem ser aplicadas, nesse caso, no entanto, usaremos listas simplesmente encadeadas circulares.
	\section{Decisões Tomadas} \label{decis}
		Não fizemos muitas coisas que ficam fora do que foi proposto no documento inicial, o que implementamos foram 3 funções extras na \verb|SparseMatrix.cpp|:
		\begin{itemize}
			\item \verb|void getHead|: \\
				retorna o atributo \verb|head|, declarado na \verb|SparseMatrix.hpp|;
			\item \verb|void getLineQty|: \\
				retorna o número de linhas da matriz;
			\item \verb|void getColQty|: \\
				retorna o número de colunas da matriz.
		\end{itemize}

		Também optamos por implementar uma \verb|main()| interativa, devidamente documentada na seção \ref{tests}.
	\section{Divisão do Trabalho} \label{div}
		A divisão do projeto foi sendo decidida no decorrer do projeto:
		
		Joabe ficou responsável pelas funções \verb|SparseMatrix()|(construtor)\verb|, insert(), print()| e \verb|readSparseMatrix()| \cite{ArquivosCFB2017, CppFilesShmeowlex2021}; e pela escrita deste documento \cite{LearnOverleaf2022}.

		Iarley ficou responsável pelas funções \verb|~SparseMatrix()|(destrutor) \verb|, get(), sum()| e \\ \verb|multiply()|; e pela \verb|main()| interativa.
	\section{Dificuldades} \label{difc}
		Dentre as dificuldades que enfrentamos estão: saber implementar cada função da matriz; pensar em cada possibilidade de erro que uma certa implementação pode gerar; saber como ligar cada nó em diferentes situações; e comunicar de forma concisa e organizada cada caso e conceito, o que pode ser confuso às vezes.
	\section{Testes Executados} \label{tests}
	\section{Análise de Complexidade} \label{complx}

	\pagebreak
	\printbibliography
\end{document}